% !TEX encoding = UTF-8
% !TEX program = pdflatex
% !TeX spellcheck = en_GB

% !TeX root = ../cv.tex
% !TeX encoding = UTF-8
% !TeX spellcheck = en_GB

% === Initial configurations ==========================
\RequirePackage{bibentry}
\makeatletter\let\saved@bibitem\@bibitem\makeatother

\documentclass[\cvLanguage, a4paper]{europasscv}

\makeatletter\let\@bibitem\saved@bibitem\makeatother

\usepackage{booktabs}
\usepackage{ifthen}

\graphicspath{{img/}}
% -----------------------------------------------------


% === Booleans settings ===============================
\newboolean{coursesBool}
\newboolean{projectsBool}
\newboolean{signatureBool}
\newboolean{urlsBool}
\newboolean{publicationsBool}
\newboolean{avatarBool}

\setboolean{coursesBool}{\enableCourses}
\setboolean{projectsBool}{\enableProjects}
\setboolean{signatureBool}{\enableSignature}
\setboolean{urlsBool}{\enableURLs}
\setboolean{publicationsBool}{\enablePublications}
\setboolean{avatarBool}{\enableAvatar}
% -----------------------------------------------------


% === Utility functions ===============================
\newcommand{\ifBoolThenElse}[3]{\ifthenelse{\boolean{#1Bool}}{#2}{#3}}%
\newcommand{\ifBoolThen}[2]{\ifBoolThenElse{#1}{#2}{}}%

\newcommand{\ifEngIta}[2]{%
  \ifthenelse{\equal{\cvLanguage}{english}}%
  {#1}%
  {%
    \ifthenelse{\equal{\cvLanguage}{italian}}%
    {#2}%
    {ERROR: WRONG LANGUAGE! Supported languages: english, italian}%
  }%
}%

\newcommand{\addURL}[1]{\ifBoolThen{urls}{ - \url{#1}}}
% -----------------------------------------------------


% === Directly defined newcommands ====================
\newcommand{\cv}[1]{%
  \begin{document}%
    \begin{europasscv}%
      #1%
    \end{europasscv}%
    \signature%
  \end{document}%
}%

\newcommand{\name}[1]{\ecvname{#1}}%
\newcommand{\address}[2]{\ecvaddress{\ifEngIta{#1}{#2}}}%
\newcommand{\cellTel}[2]{\ecvtelephone[#1]{#2}}%
\newcommand{\email}[1]{\ecvemail{#1}}%
\newcommand{\homepage}[1]{\ecvhomepage{#1}}%
\newcommand{\instantMessaging}[2]{\ecvim{#1}{#2}}%
\newcommand{\gender}[2]{\ecvgender{\ifEngIta{#1}{#2}}}%
\newcommand{\dateOfBirth}[2]{\ecvdateofbirth{\ifEngIta{#1}{#2}}}%
\newcommand{\nationality}[2]{\ecvnationality{\ifEngIta{#1}{#2}}}%
\newcommand{\avatar}[1]{\ifBoolThen{avatar}{\ecvpicture[width=3.8cm]{#1}}}%

\newcommand{\personalInfo}[1]{%
  #1
  \ecvpersonalinfo%
}%

\newcommand{\motherTongue}[2]{\ecvmothertongue{\ifEngIta{#1}{#2}}}%
\newcommand{\otherLanguage}[7]{\ecvlanguage{\ifEngIta{#1}{#2}}{#3}{#4}{#5}{#6}{#7}}%
\newcommand{\lastLanguage}[7]{\ecvlastlanguage{\ifEngIta{#1}{#2}}{#3}{#4}{#5}{#6}{#7}}%
\newcommand{\otherLanguages}[1]{\ecvlanguageheader#1\ecvlanguagefooter}%

\newcommand{\listSection}[1]{\begin{ecvitemize}#1\end{ecvitemize}}%
\newcommand{\listItem}[1]{\item #1}%

\newcommand{\signature}{%
  \ifBoolThen{signature}{%
    \null\vspace*{\stretch{1}}%
    \flushright%
    \begin{tabular}{rp{250pt}}%
      \ifEngIta{%
        & \small My personal details can be processed into your data system.\\%
      }{%
        & \small Autorizzo il trattamento dei dati personali contenuti nel mio curriculum vitae in base art. 13 del \href{http://www.garanteprivacy.it/garante/doc.jsp?ID=1311248}{D. Lgs. 196/2003}.\\%
      }%
      & \\%
      \ifEngIta{%
        & Signature:\\%
      }{%
        & Firma:\\%
      }%
      & \\%
      \bottomrule%
    \end{tabular}%
  }%
}%

\newcommand{\publication}[1]{\listItem{\bibentry{#1}}}%

\newcommand{\customSection}[4]{%
  \ecvblueitem{\ifEngIta{#1}{#3}}{\ifEngIta{#2}{#4}}%
}%
% -----------------------------------------------------


% === Meta-newcommand functions definition ==========
\newcommand{\stringCommand}[3]{%
  \ifEngIta{%
    \expandafter\newcommand\csname #1String\endcsname{#2}%
  }{%
    \expandafter\newcommand\csname #1String\endcsname{#3}%
  }%
}%

\newcommand{\bigItemCommand}[1]{%
  \expandafter\newcommand\csname #1\endcsname[2]{%
    \ecvbigitem{\csname #1String\endcsname}{\ifEngIta{##1}{##2}}%
  }%
}%

\newcommand{\sectionCommand}[1]{%
  \expandafter\newcommand\csname #1\endcsname{%
    \ecvsection{\csname #1String\endcsname}%
  }%
}%

\newcommand{\titleCommand}[1]{%
  \expandafter\newcommand\csname #1\endcsname[4]{%
    \ecvtitle{\ifEngIta{##1}{##3}}{\ifEngIta{##2}{##4}}%
  }%
}%
\newcommand{\titleLevelCommand}[1]{%
  \expandafter\newcommand\csname #1\endcsname[5]{%
    \ecvtitlelevel{\ifEngIta{##1}{##3}}{\ifEngIta{##2}{##4}}{##5}%
  }%
}%

\newcommand{\ecvItemCommand}[1]{%
  \expandafter\newcommand\csname #1\endcsname[2]{%
    \ecvitem{}{\ifEngIta{##1}{##2}}%
  }%
}%
\newcommand{\ecvItemTitleCommand}[1]{%
  \expandafter\newcommand\csname #1\endcsname[2]{%
    \ecvitem{\csname #1String\endcsname}{\ifEngIta{##1}{##2}}%
  }%
}%
\newcommand{\ecvItemTitleURLCommand}[1]{%
  \expandafter\newcommand\csname #1URL\endcsname[3]{%
    \ecvitem{\csname #1String\endcsname}{\ifEngIta{##1}{##2}\addURL{##3}}%
  }%
}%
\newcommand{\ecvItemIfTitleCommand}[2]{%
  \expandafter\newcommand\csname #1\endcsname[2]{%
    \ifBoolThen{#2}{\ecvitem{\csname #1String\endcsname}{\ifEngIta{##1}{##2}}}%
  }%
}%
\newcommand{\ecvItemIfTitleListCommand}[2]{%
  \expandafter\newcommand\csname #1\endcsname[1]{%
    \ifBoolThen{#2}{\ecvitem{\csname #1String\endcsname}{\begin{ecvitemize}##1\end{ecvitemize}}}%
  }%
}%

\newcommand{\itemTitleCommand}[1]{%
  \expandafter\newcommand\csname #1\endcsname[3]{%
    \listItem{\textit{##1}: \ifEngIta{##2}{##3}}%
  }%
}%
\newcommand{\itemTitleURLCommand}[1]{%
  \expandafter\newcommand\csname #1URL\endcsname[4]{%
    \csname #1\endcsname{##1}{##2}{##3}\addURL{##4}%
  }%
}%

\newcommand{\blueItemCommand}[1]{%
  \expandafter\newcommand\csname #1\endcsname[2]{%
    \ecvblueitem{\csname #1String\endcsname}{\ifEngIta{##1}{##2}}%
  }%
}%
\newcommand{\blueItemListCommand}[1]{%
  \expandafter\newcommand\csname #1\endcsname[1]{%
    \ecvblueitem{\csname #1String\endcsname}{\begin{ecvitemize}##1\end{ecvitemize}}%
  }%
}%
\newcommand{\blueItemIfListCommand}[2]{%
  \expandafter\newcommand\csname #1\endcsname[1]{%
    \ifBoolThen{#2}{\ecvblueitem{\csname #1String\endcsname}{\begin{ecvitemize}##1\end{ecvitemize}}}%
  }%
}%
\newcommand{\blueItemIfListExtraCommands}[3]{%
  \expandafter\newcommand\csname #1\endcsname[1]{%
    #3
    \ifBoolThen{#2}{\ecvblueitem{\csname #1String\endcsname}{\begin{ecvitemize}##1\end{ecvitemize}}}%
  }%
}%

\newcommand{\itemListCommand}[1]{%
  \expandafter\newcommand\csname #1\endcsname[2]{%
    \listItem{\ifEngIta{##1}{##2}}%
  }%
}%
% -----------------------------------------------------


% === Meta-newcommands creation ====================
\stringCommand{jobAppliedFor}{Job applied for}{Impiego ricercato}
\stringCommand{workExperience}{Work experience}{Esperienza professionale}
\stringCommand{educationAndTraining}{Education and training}{Istruzione e formazione}
\stringCommand{thesis}{Thesis}{Tesi}
\stringCommand{courses}{Courses}{Corsi}
\stringCommand{projects}{Projects}{Progetti}
\stringCommand{finalRank}{Final rank}{Voto finale}
\stringCommand{personalSkills}{Personal skills}{Capacità personali}
\stringCommand{communicationSkills}{Communication skills}{Capacità di comunicazione}
\stringCommand{organisationalSkills}{Organisational / managerial skills}{Capacità organizzative / manageriali}
\stringCommand{computerSkills}{Computer skills}{Capacità informatiche}
\stringCommand{otherSkills}{Other skills}{Altre capacità}
\stringCommand{drivingLicence}{Driving licence}{Patente}
\stringCommand{additionalInfo}{Additional information}{Ulteriori informazioni}
\stringCommand{references}{References}{Referenze}
\stringCommand{otherProjects}{Other projects}{Altri progetti}
\stringCommand{publications}{Publications}{Pubblicazioni}

\bigItemCommand{jobAppliedFor}

\sectionCommand{workExperience}
\sectionCommand{educationAndTraining}
\sectionCommand{personalSkills}
\sectionCommand{additionalInfo}

\titleCommand{job}
\titleLevelCommand{school}

\ecvItemCommand{jobLocation}
\ecvItemCommand{jobDepartment}
\ecvItemCommand{jobDescription}
\ecvItemCommand{schoolLocation}
\ecvItemCommand{schoolFaculty}
\ecvItemCommand{schoolDescription}
\ecvItemTitleCommand{thesis}
\ecvItemTitleCommand{finalRank}
\ecvItemTitleURLCommand{thesis}
\ecvItemIfTitleCommand{courses}{courses}
\ecvItemIfTitleListCommand{projects}{projects}

\itemTitleCommand{project}
\itemTitleURLCommand{project}

\blueItemCommand{communicationSkills}
\blueItemCommand{organisationalSkills}
\blueItemCommand{computerSkills}
\blueItemCommand{otherSkills}
\blueItemCommand{drivingLicence}
\blueItemListCommand{references}
\blueItemIfListCommand{otherProjects}{projects}
\blueItemIfListExtraCommands{publications}{publications}{\bibliographystyle{plain}\nobibliography{back/publications}}

\itemListCommand{reference}
% -----------------------------------------------------


\ecvname{Tommaso Papini}
\ecvaddress{33, Via Poggio alla Croce, 50063, Figline e Incisa Valdarno (FI), Italia}
\ecvtelephone[+39 3343621205]{+39 0559500024}
\ecvemail{tommaso.papini@unifi.it}
\ecvhomepage{https://www.linkedin.com/in/tommaso-papini/}
\ecvim{Skype}{tommy39ita}
\ecvdateofbirth{30 Settembre 1990}
\ecvnationality{Italiana}
\ecvgender{Maschile}

\graphicspath{{img/}}

\ecvpicture[width=3.8cm]{avatar.jpg}

\begin{document}
  \begin{europasscv}
  
  \ecvpersonalinfo
  
  \ecvbigitem{Impiego ricercato}{Sviluppo e Ricerca Software}
  
  %------------------------------------------------------
  \ecvsection{Esperienza professionale}
  %------------------------------------------------------
  
    \ecvtitle{Giugno 2016 -- Ott 2016}{Assegnista di Ricerca}
      \ecvitem{}{Università degli Studi di Firenze (Italia)}
      \ecvitem{}{Facoltà d'Ingegneria, Dipartimento d'Ingegneria dell'Informazione}
      \ecvitem{}{Analisi quantitativa basata su modelli per sistemi non Markoviani}
    
    \ecvtitle{Ott 2013 -- Nov 2014}{Tirocinio come Technical Student}
      \ecvitem{}{CERN, Route de Meyrin 385, 1217 Meyrin (Svizzera)}
      \ecvitem{}{Sviluppatore web per l'Indico Knowledge Transfer Project, un progetto mirato al miglioramento dell'impatto a livello mondiale di Indico. Indico è un'applicazione web per l'organizzazione di eventi}
  
  %------------------------------------------------------
  \ecvsection{Istruzione e formazione}
  %------------------------------------------------------
  
    \ecvtitlelevel{Nov 2016 -- Presente}{Dottorato in Smart Computing}{ISCED 8}
      \ecvitem{}{Università degli Studi di Firenze, Pisa e Siena (Italia)}
      \ecvitem{}{Facoltà d'Ingegneria, Dipartimento d'Ingegneria dell'Informazione}
      \ifCourses{\ecvitem{Corsi}{}}
      \ecvitem{Tesi}{Analisi quantitativa basata su modelli per diagnosi, predizione, schedulazione e valutazione di aderenza on-line per sistemi parzialmente osservabili \ifURLs{https://github.com/oddlord/phd-sc-research-project/raw/master/research_project.pdf}}
    
    \ecvtitlelevel{Dic 2012 -- Apr 2016}{Laurea Magistrale in Informatica}{ISCED 7}
      \ecvitem{}{Università degli Studi di Firenze (Italia)}
      \ecvitem{}{Facoltà di Scienze Matematiche, Fisiche e Naturali}
      \ifCourses{\ecvitem{Corsi}{Progettazione e Analisi di Algoritmi, Data Warehousing, Architetture Avanzate, Teoria dei Linguaggi di Programmazione, Analisi Quantitativa dei Sistemi, Complementi di Calcolo Numerico, Modelli di Sistemi Sequenziali e Concorrenti, Interazione Uomo Macchina, Documentazione Automatica, Apprendimento Automatico, Metodi di Verifica e Testing, Data Mining, Teoria dell'Informazione}}
      \ifProjects{\ecvitem{Progetti}{
        \begin{ecvitemize}
          \item \textit{Rankboost}: Implementazione in C++ dell'algoritmo learning-to-rank Rankboost. Incluso nel software Quickrank sviluppato da HCP Lab, ISTI, CNR \ifURLs{https://github.com/hpclab/quickrank}
          \item \textit{MRP steady-state}: Implementazione in Java di un algoritmo per il calcolo dello steady-state per Processi rigenerativi di Markov. Incluso nel software Oris sviluppato da STLab, Università degli Studi di Firenze \ifURLs{https://github.com/trianam/mvt/raw/master/steady_state_MRP.pdf}
        \end{ecvitemize}
      }}
      \ecvitem{Tesi}{
        Il progetto Indico KT: Migliorare l'impatto a livello mondiale di Indico \ifURLs{https://github.com/oddlord/tesi-magistrale/raw/master/tesi/tesi.pdf}
      }
      \ecvitem{Voto finale}{110/110 con lode}
    
    \ecvtitlelevel{Sett 2010 -- Luglio 2011}{Laurea Triennale in Ingegneria Informatica (Erasmus)}{ISCED 6}
      \ecvitem{}{Università Politecnica di Madrid (Spagna)}
      \ecvitem{}{Facoltà di Informatica}
      \ifCourses{\ecvitem{Corsi}{Fondamenti Fisici e Tecnologici dell'Informatica, Algebra Lineare, Probabilità e Statistica I, Programmazione II, Sicurezza Informatica, Programmazione di Sistemi, Sistemi Operativi, Basi di Dati}}
    
    \ecvtitlelevel{Ott 2009 -- Dic 2012}{Laurea Triennale in Informatica}{ISCED 6}
      \ecvitem{}{Università degli Studi di Firenze (Italia)}
      \ecvitem{}{Facoltà di Scienze Matematiche, Fisiche e Naturali}
      \ifCourses{\ecvitem{Corsi}{Inglese, Algoritmi e Strutture Dati, Analisi I \& II, Architetture degli Elaboratori, Matematica Discreta e Logica, Programmazione, Basi di Dati e Sistemi Informativi, Sistemi Operativi, Fisica Generale, Calcolo delle Probabilità e Statistica, Programmazione Concorrente, Metodologie di Programmazione, Algebra Lineare, Competenze Aziendali, Modelli e Calcoli per la Fisica, Intelligenza Artificiale, Codici e Sicurezza, Informatica Teorica, Calcolo Numerico, Reti di Calcolatori}}
      \ifProjects{\ecvitem{Progetti}{
        \begin{ecvitemize}
          \item \textit{Cerithidea Decollata}: rete neurale che simula la predizione delle maree di alcune chiocciole intertidali \ifURLs{https://github.com/oddlord/cerithidea-decollata-model/raw/master/CerithideaModel.pdf}
        \end{ecvitemize}
      }}
      \ecvitem{Tesi}{
        Visualizzazione grafica di algoritmi in HTML5 \ifURLs{https://github.com/oddlord/tesi-triennale/raw/master/tesi.pdf}
      }
      \ecvitem{Voto finale}{110/110 con lode}
    
    \ecvtitlelevel{Sett 2004 -- Luglio 2009}{Liceo Scientifico}{ISCED 3}
      \ecvitem{}{Istituto Statale di Istruzione Superiore \textit{Giorgio Vasari}, Figline e Incisa Valdarno (Italia)}
      \ecvitem{}{Piano Nazionale di Informatica (PNI)}
      \ecvitem{Voto finale}{76/100}
  
  %------------------------------------------------------
  \ecvsection{Capacità personali}
  %------------------------------------------------------
  
    \ecvmothertongue{Italiano}
    \ecvlanguageheader
    \ecvlanguage{Inglese}{B2}{C1}{B2}{C1}{C1}
    \ecvlastlanguage{Spagnolo}{B2}{C1}{B2}{C1}{C1}
    \ecvlanguagefooter
    
    \ecvblueitem{Capacità di comunicazione}{Ho lavorato in gruppi sia di sviluppo che di ricerca}
    
    \ecvblueitem{Capacità organizzative / manageriali}{Durante il Dottorato e l'attività di ricerca ho seguito diversi studenti sia per progetti d'esame che per ricerche di tesi}
    
    \ecvblueitem{Capacità informatiche}{
      \begin{ecvitemize}
        \item linguaggi di programmazione: Java, C, C++, shell scripting, assembler MIPS, LaTeX, Matlab, HTML5, XML, programmazione dichiarativa, SQL, $\lambda$-calcolo, Python, Javascript, CSS 
        \item strumenti \& software: Jinja2, Git, Github, QEMU, VirtualBox, SASS, Eclipse, Windows, Linux, dotfiles, Python Fabric, Jekyll
      \end{ecvitemize}
    }
    
    \ecvblueitem{Altre capacità}{Suonare la chitarra elettrica ed ascoltare musica. Allenarsi come pugile e fare sport in generale. Leggere libri di vario genere. Grande passione per le scienze. Viaggiare per il mondo e scoprire nuove culture}
    
    \ecvblueitem{Patente}{B (automunito)}
  
  %------------------------------------------------------
  \ecvsection{Ulteriori informazioni}
  %------------------------------------------------------
  
    \ecvblueitem{Referenze}{
      \begin{ecvitemize}
        \item Prof. Enrico Vicario, Università degli Studi di Firenze (Italia) 
        \item Prof. Pierluigi Crescenzi, Università degli Studi di Firenze (Italia)
        \item Prof. Gregorio Landi, Università degli Studi di Firenze (Italia)
        \item Pedro Ferreira, CERN, Ginevra (Svizzera)
      \end{ecvitemize}
    }
    
    \ifProjects{\ecvblueitem{Altri progetti}{
      \begin{ecvitemize}
        \item \textit{Blindstore}: data store per il recupero di dati anonimo. Vincitore del Premio Tecnologia al CERN Summer Student Webfest 2014 \& concorrente dell'Hackathon The Port 2014 @CERN \ifURLs{http://blindstore.github.io/}
      \end{ecvitemize}
    }}
    
    \ifPublications{
  	  \bibliographystyle{plain}
      \nobibliography{publications}
      
      \ecvblueitem{Pubblicazioni}{
        \begin{ecvitemize}
          \item \bibentry{martina2016performance}
        \end{ecvitemize}
      }
    }
  
  \end{europasscv}
  
  \ifSignature{
    \null\vspace*{\stretch{1}}
    \flushright
    \begin{tabular}{rp{250pt}}
      & \small Autorizzo il trattamento dei dati personali contenuti nel mio curriculum vitae in base art. 13 del \url{http://www.garanteprivacy.it/garante/doc.jsp?ID=1311248}{D. Lgs. 196/2003}.\\
      & \\
      & Firma:\\
      & \\
      \bottomrule
    \end{tabular}
  }
  
\end{document}