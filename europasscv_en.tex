% !TEX encoding = UTF-8
% !TEX program = pdflatex
% !TEX spellcheck = en_GB

\documentclass[english,a4paper]{europasscv}

\ecvname{Tommaso Papini}
\ecvaddress{33, Via Poggio alla Croce, 50063, Figline e Incisa Valdarno (FI), Italy}
\ecvtelephone[+39 3343621205]{+39 0559500024}
\ecvemail{tommaso.papini@unifi.it}
\ecvhomepage{https://www.linkedin.com/in/tommaso-papini/}
\ecvim{Skype}{tommy39ita}

\ecvdateofbirth{30 September 1990}
\ecvnationality{Italian}
\ecvgender{Male}

\ecvpicture[width=3.8cm]{img/avatar.jpg}

\begin{document}
  \begin{europasscv}

  \ecvpersonalinfo

  \ecvbigitem{Job applied for}{Software Development \& Research}

  %------------------------------------------------------
  \ecvsection{Work experience}
  %------------------------------------------------------
  
  \ecvtitle{June 2016 -- Oct 2016}{Research Fellow}
  \ecvitem{}{University of Florence, Department of Computer Science \newline Via di S. Marta 3, 50139, Florence (Italy)}
  \ecvitem{}{Model based quantitative analysis for non-Markovian systems}
  
  \ecvtitle{Oct 2013 -- Nov 2014}{Technical Student Internship}
  \ecvitem{}{CERN, Route de Meyrin 385, 1217 Meyrin (Switzerland)}
  \ecvitem{}{Web developer for the Indico Knowledge Transfer Project, a project aimed to increased the worldwide impact of Indico, a web-application for event organization}
% \ecvitem{}{More info: \url{https://github.com/oddlord/tesi-magistrale/raw/master/tesi/tesi.pdf}.}
  
  %------------------------------------------------------
  \ecvsection{Education and training}
  %------------------------------------------------------
  
  \ecvtitlelevel{Nov 2016 -- Present}{PhD in Smart Computing}{ISCED 8}
  \ecvitem{}{Universities of Florence, Pisa and Siena (Italy)}
  \ecvitem{}{Faculty of Engineering, Department of Information Engineering}
% \ecvitem{Courses}{}
  \ecvitem{Thesis}{Model-based quantitative analysis for on-line diagnosis, prediction, scheduling and compliance evaluation in partially observable systems}
  
  \ecvtitlelevel{Dec 2012 -- Apr 2016}{Master in Computer Science}{ISCED 7}
  \ecvitem{}{University of Florence (Italy)}
  \ecvitem{}{Faculty of Maths, Physics and Natural Sciences}
% \ecvitem{Courses}{Design and Analysis of Algorithms, Data Warehousing, Non-standard Architectures, Foundations of Programming Languages, Quantitative Analysis of Systems, Advanced Numerical Analysis, Models of Sequential and Concurrent Systems, Human Computer Interaction, Information Retrieval and Semantic Web Technologies, Machine Learning, Verification and Testing Methods, Data Mining, Information Theory}
  \ecvitem{Thesis}{The Indico KT Project: Improving the worldwide impact of Indico}
  \ecvitem{Final Rank}{110/110 cum laude}
  
  \ecvtitlelevel{Sept 2010 -- July 2011}{Bachelor in Computer Engineering (Erasmus)}{ISCED 6}
  \ecvitem{}{Polytechnic University of Madrid (Spain)}
  \ecvitem{}{Faculty of Computer Science}
% \ecvitem{Courses}{Physical and Technological Foundations of Informatics, Linear Algebra, Probability and Statistics I, Programming II, Information Technology Security, Systems Programming, Operating Systems, Databases}
  
  \ecvtitlelevel{Oct 2009 -- Dec 2012}{Bachelor in Computer Science}{ISCED 6}
  \ecvitem{}{University of Florence (Italy)}
  \ecvitem{}{Faculty of Maths, Physics and Natural Sciences}
% \ecvitem{Courses}{English, Algorithm and Data Structures, Real Analysis I \& II, Computer Architecture, Discrete Mathematics and Mathematical Logics, Programming, Data Bases and Information Systems, Operating Systems, Physics, Probability Theory and Statistics, Concurrent Programming, Programming Methodologies, Linear Algebra, Computing and Management, Models and Calculi for Physics, Artificial Intelligence, Coding Theory and Computer Security, Theoretical Computer Science, Numerical Analysis, Computer Networks}
  \ecvitem{Thesis}{Algorithm Visualization in HTML5}
  \ecvitem{Final Rank}{110/110 cum laude}
  
  \ecvtitlelevel{Sept 2004 -- July 2009}{Scientific High School}{ISCED 3}
  \ecvitem{}{State Institute of Higher Education \textit{Giorgio Vasari}, Figline e Incisa Valdarno (Italy)}
  \ecvitem{}{National Plan of Computer Studies (PNI)}
  \ecvitem{Final rank}{76/100}
  
%   \pagebreak
  
  %------------------------------------------------------
  \ecvsection{Personal skills}
  %------------------------------------------------------
  
  \ecvmothertongue{Italian}
  \ecvlanguageheader
  \ecvlanguage{English}{C1}{C1}{C1}{C1}{C1}
  \ecvlastlanguage{Spanish}{C1}{C1}{C1}{C1}{C1}
  \ecvlanguagefooter
   
  \ecvblueitem{Communication skills}{
  \begin{ecvitemize}
    \item team work: I have worked in various types of teams from research teams to national league hockey. For 2 years I coached my university hockey team
    \item mediating skills: I work on the borders between young people, youth trainers, youth policy and researchers, for example running a 3 day workshop at CoE Symposium ``Youth Actor of Social Change'', and my continued work on youth training programmes 
    \item intercultural skills: I am experienced at working in a European dimension such as being a rapporteur at the CoE Budapest ``youth against violence seminar'' and working with refugees.
  \end{ecvitemize}
  }
  
  \ecvblueitem{Organisational / managerial skills}{
  \begin{ecvitemize}
    \item whilst working for a Brussels based refugee NGO ``Convivial'' I organized a ``Civil Dialogue'' between refugees and civil servants at the European Commission 20th June 2002
    \item during my PhD I organised a seminar series on research methods
  \end{ecvitemize}
  }
  
  \ecvblueitem{Computer skills}{
  \begin{ecvitemize}
    \item competent with most Microsoft Office programmes
    \item experience with HTML
  \end{ecvitemize}
  }
  
  
  \ecvblueitem{Other skills}{Creating pieces of Art and visiting Modern Art galleries. Enjoy all sports particularly hockey, football and running. Love to travel and experience different cultures.}

  \ecvblueitem{Driving licence}{A, B}
  
  \ecvsection{Additional information}
  
  \ecvblueitem{Publications}{\textit{How to do Observations: Borrowing techniques from the Social Sciences to help Participants do Observations in Simulation Exercises}, Coyote EU/CoE Partnership Publication, (2002).
}
  
  \end{europasscv}

\end{document}