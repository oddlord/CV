\documentclass[totpages, helvetica, openbib, italian]{europecv}

%--------------------------------------------------------------

\usepackage[T1]{fontenc}
\usepackage[italian]{babel}
\usepackage[a4paper, top=1.27cm, left=1cm, right=1cm, bottom=2cm]{geometry}
\usepackage{bibentry, url, pdfpages, graphicx, xcolor, booktabs}
\usepackage[urlcolor=blue, colorlinks=true]{hyperref}

%--------------------------------------------------------------

\renewcommand{\ttdefault}{phv} % Uses Helvetica instead of fixed width font

\ecvname{Papini Tommaso}
\ecvfootername{Tommaso Papini}
\ecvaddress{33, Via Poggio alla Croce, 50063, Figline Valdarno (FI), Italia}
\ecvtelephone[+39 334/3621205]{+39 055/9500024}
%\ecvfax{}
\ecvemail{\url{tommy39@gmail.com}}
\ecvnationality{Italiana}
\ecvdateofbirth{30/09/1990}
\ecvgender{Maschile}
\ecvpicture[width=3cm]{avatar.jpg}
\ecvfootnote{Per ulteriori informazioni: \url{https://europass.cedefop.europa.eu}\\
\textcopyright~European Union, 2002-2016.}

\graphicspath{{img/}}

%--------------------------------------------------------------

\begin{document}
	
	\selectlanguage{italian}
	
	\begin{europecv}
		\ecvpersonalinfo[20pt]
		
		\ecvitem{\large\textbf{Impiego ricercato/ Settore di competenza}}{\large\textbf{Sviluppo e ricerca software}}

		\ecvsection{Esperienza professionale}
		\ecvitem{Date}{01/10/2013 - 30/11/2014.}
		\ecvitem{Funzione o posto occupato}{Technical Student.}
		\ecvitem{Principali mansioni e responsabilità}{Sviluppatore Web per il software Indico, progetto finanziato dal gruppo Knowledge Transfer al CERN.}
		\ecvitem{Nome e indirizzo del datore di lavoro}{CERN, Route de Meyrin 385 1217 Meyrin, Svizzera.}
		\ecvitem{Tipo o settore d'attività}{Sviluppo Web.}
		
%--------------------------------------------------------------
		
		\ecvsection{Istruzione e formazione}
		
		\ecvitem{Date}{19/12/2012 - in corso.}
		\ecvitem{Principali materie/Competenze professionali apprese}{Data Warehousing, Teoria dei Linguaggi di Programmazione, Analisi Quantitativa dei Sistemi, Teoria dell'Informazione, Interazione Uomo Macchina, Complementi di Calcolo Numerico, Documentazione Automatica, Architetture Avanzate, Metodi di Verifica e Testing, Progettazione e Analisi degli Algoritmi, Data Mining, Apprendimento Automatico, Modelli di Sistemi Sequenziali e Concorrenti.}
		\ecvitem[20pt]{Nome e tipo d'istituto di istruzione o formazione}{Università degli Studi di Firenze, Facoltà di Scienze Matematiche, Fisiche e Naturali, Corso di Laurea Magistrale in Informatica.}
		
		\ecvitem{Date}{06/10/2009 - 14/12/2012.}
		\ecvitem{Certificato o diploma ottenuto}{Laurea in Informatica.}
		\ecvitem{Principali materie/Competenze professionali apprese}{Matematica, Architetture degli Elaboratori, Algoritmi e Strutture Dati, Programmazione (di Base e Concorrente), Fisica, Basi di Dati, Informatica Teorica, IA, Reti di Calcolatori, Reti Neurali, Analisi Numerica.}
		\ecvitem{Nome e tipo d'istituto di istruzione o formazione}{Università degli Studi di Firenze, Facoltà di Scienze Matematiche, Fisiche e Naturali, Corso di Laurea Triennale in Informatica.}
		\ecvitem[20pt]{Livello nella classificazione nazionale o internazionale}{110/110 con lode.}
		
		\ecvitem{Date}{Settembre 2010 - Luglio 2011.}
		\ecvitem{Principali materie/Competenze professionali apprese}{Matematica, Statistica, Fisica, Codici e Sicurezza, Pattern di Programmazione, Basi di Dati, Sistemi Operativi.}
		\ecvitem[20pt]{Nome e tipo d'istituto di istruzione o formazione}{Università Politecnica di Madrid, Facoltà di Informatica, Ingegneria Informatica (Programma Erasmus).}
		
		\ecvitem{Date}{Settembre 2004 - Luglio 2009.}
		\ecvitem{Certificato o diploma ottenuto}{Diploma di istruzione superiore.}
		\ecvitem{Nome e tipo d'istituto di istruzione o formazione}{Istituto Statale di Istruzione Superiore Giorgio Vasari, Liceo Scientifico sperimentale (PNI).}
		\ecvitem[20pt]{Livello nella classificazione nazionale o internazionale}{76/100.}
		
%--------------------------------------------------------------
		
		\ecvsection{Capacità e competenze professionali}
		
		\ecvmothertongue[5pt]{Italiano}
		\ecvitem{\large Altra/e lingua/e}{}
		\ecvlanguageheader{(*)}
		\ecvlanguage{Inglese}{\ecvBOne}{\ecvBOne}{\ecvBOne}{\ecvBOne}{\ecvCOne}
		\ecvlastlanguage{Spagnolo}{\ecvCOne}{\ecvCOne}{\ecvCOne}{\ecvCOne}{\ecvCTwo}
		\ecvlanguagefooter[10pt]{(*)}
		
%--------------------------------------------------------------
		
		\ecvitem[10pt]{\large Capacità e competenze sociali}{Volontario della Croce Rossa Italiana.}

%		\ecvitem[10pt]{\large Capacità e competenze organizzative}{Descrivere tali competenze e indicare dove sono state acquisite. Facoltativo.}

		\ecvitem[10pt]{\large Capacità e competenze tecniche}{Discrete capacità tecniche, specialmente in campo informatico ed elettronico, apprese per la maggior parte come autodidatta.}

		\ecvitem[10pt]{\large Capacità e competenze informatiche}{Grandi capacità informatiche, apprese sia come autodidatta che tramite l'Università. Grandi conoscenze nell'utilizzo dei più svariati linguaggi di programmazione/markup (tra i meglio conosciuti: Java, C, Shell Scripting, Assembler, LaTeX, Matlab, HTML5, XML, Programmazione Dichiarativa, SQL, $\lambda$-calcolo, Python, Javascript, CSS). Buona familiarità con molti strumenti come templates Jinja2, Guthub, strumenti di virtualizzazione, SASS. Alto grado di maneggevolezza sia per aspetti software che hardware.}

		\ecvitem[10pt]{\large Capacità e competenze artistiche}{Discreto livello nell'utilizzo della chitarra e nel canto (amatoriale).}

%		\ecvitem[10pt]{\large Altre capacità e competenze}{Descrivere tali competenze e indicare dove sono state acquisite. Facoltativo.}

		\ecvitem{\large Patente/i}{Patente di guida (Repubblica Italiana, piena validità Europea), categoria B (automunito).}
		
%--------------------------------------------------------------

		\ecvsection{Ulteriori informazioni}
%		\ecvitem[10pt]{}{Inserire qui ogni altra informazione utile, ad esempio persone di riferimento, referenze, etc\ldots Facoltativo.}
		\ecvitem{}{\textbf{Referenze}}
		\ecvitem{}{Prof. Pierluigi Crescenzi, Università degli Studi di Firenze, Italia;}
		\ecvitem{}{Prof. Gregorio Landi, Università degli Studi di Firenze, Italia.}
		\ecvitem[15pt]{}{Pedro Ferreira, CERN, Ginevra.}

		\ecvitem{}{\textbf{Altri progetti}}
		\ecvitem{}{\textit{Rankboost}: Implementazione in C++ dell'algoritmo learning-to-rank Rankboost. \url{https://github.com/hpclab/quickrank}.}
		\ecvitem{}{\textit{Blindstore}: vincitore del Premio Tecnologia al CERN Summer Student Webfest 2014 \& participante dell'Hackathon The Port 2014 @CERN. \url{http://blindstore.github.io/}.}
		\ecvitem{}{\textit{Cerithidea Decollata}: rete neurale che simula la predizione delle maree di alcune chiocciole intertidali. \url{https://github.com/oddlord/cerithidea-decollata-model}.}
		\ecvitem[10pt]{}{\textit{Visualizzazione di Algoritmi in HTML5}. \url{https://github.com/oddlord/tesi-triennale}.}

%		\bibliographystyle{plain}
%		\nobibliography{publications}
%		\ecvitem{}{\textbf{Pubblicazioni}}
%		\ecvitem{}{\bibentry{pub1}}
%		\ecvitem[10pt]{}{\bibentry{pub2}}

		\ecvitem{}{\textbf{Interessi personali}}
		\ecvitem{}{Informatica, Tecnologia, Matematica, Fisica e la Scienza in generale. Libri, Musica e Spettacolo. Sport di qualsiasi tipo (al momento pugile e ciclista amatoriale). Viaggiare per il mondo.}
		
%		\ecvsection{Allegati}
%		\ecvitem{}{Certificato di Laurea Triennale in Informatica con elenco completo degli esami sostenuti (copia);}
%		\ecvitem[5pt]{}{Relazione Informativa di Supplemento al Diploma (copia).}

%		\ecvitem{}{Attestazione del programma di Techical Student al CERN (copia, in inglese);}
%		\ecvitem{}{Attestazione di partecipazione al CERN Summer Webfest 2014 (copia,  in inglese);}
%		\ecvitem{}{Attestazione di partecipazione al The Port Hackathon 2014, al CERN (copia,  in inglese).}

	\end{europecv}


	\null\vspace*{\stretch{1}}
	\flushright
	\begin{tabular}{rp{250pt}}
	 & \small Autorizzo il trattamento dei dati personali contenuti nel mio curriculum vitae in base art. 13 del \href{http://www.garanteprivacy.it/garante/doc.jsp?ID=1311248}{D. Lgs. 196/2003}.\\ 
	 &  \\ 
	 & Firma:\\ 
	 &  \\
	  \bottomrule
	\end{tabular}


\end{document} 